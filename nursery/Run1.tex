% Options for packages loaded elsewhere
\PassOptionsToPackage{unicode}{hyperref}
\PassOptionsToPackage{hyphens}{url}
%
\documentclass[
]{article}
\usepackage{amsmath,amssymb}
\usepackage{iftex}
\ifPDFTeX
  \usepackage[T1]{fontenc}
  \usepackage[utf8]{inputenc}
  \usepackage{textcomp} % provide euro and other symbols
\else % if luatex or xetex
  \usepackage{unicode-math} % this also loads fontspec
  \defaultfontfeatures{Scale=MatchLowercase}
  \defaultfontfeatures[\rmfamily]{Ligatures=TeX,Scale=1}
\fi
\usepackage{lmodern}
\ifPDFTeX\else
  % xetex/luatex font selection
\fi
% Use upquote if available, for straight quotes in verbatim environments
\IfFileExists{upquote.sty}{\usepackage{upquote}}{}
\IfFileExists{microtype.sty}{% use microtype if available
  \usepackage[]{microtype}
  \UseMicrotypeSet[protrusion]{basicmath} % disable protrusion for tt fonts
}{}
\makeatletter
\@ifundefined{KOMAClassName}{% if non-KOMA class
  \IfFileExists{parskip.sty}{%
    \usepackage{parskip}
  }{% else
    \setlength{\parindent}{0pt}
    \setlength{\parskip}{6pt plus 2pt minus 1pt}}
}{% if KOMA class
  \KOMAoptions{parskip=half}}
\makeatother
\usepackage{xcolor}
\usepackage[margin=1in]{geometry}
\usepackage{graphicx}
\makeatletter
\def\maxwidth{\ifdim\Gin@nat@width>\linewidth\linewidth\else\Gin@nat@width\fi}
\def\maxheight{\ifdim\Gin@nat@height>\textheight\textheight\else\Gin@nat@height\fi}
\makeatother
% Scale images if necessary, so that they will not overflow the page
% margins by default, and it is still possible to overwrite the defaults
% using explicit options in \includegraphics[width, height, ...]{}
\setkeys{Gin}{width=\maxwidth,height=\maxheight,keepaspectratio}
% Set default figure placement to htbp
\makeatletter
\def\fps@figure{htbp}
\makeatother
\setlength{\emergencystretch}{3em} % prevent overfull lines
\providecommand{\tightlist}{%
  \setlength{\itemsep}{0pt}\setlength{\parskip}{0pt}}
\setcounter{secnumdepth}{-\maxdimen} % remove section numbering
\ifLuaTeX
  \usepackage{selnolig}  % disable illegal ligatures
\fi
\usepackage{bookmark}
\IfFileExists{xurl.sty}{\usepackage{xurl}}{} % add URL line breaks if available
\urlstyle{same}
\hypersetup{
  pdftitle={Elsoms Run 1},
  pdfauthor={Petra Guy},
  hidelinks,
  pdfcreator={LaTeX via pandoc}}

\title{Elsoms Run 1}
\author{Petra Guy}
\date{2025-11-17}

\begin{document}
\maketitle

This is data analysis notes - more details also see Nursery
Trials/Experiments/Run 1/Elsoms Run 1 Results

We grew sitka, birch and pine with different emf at different fertilizer
levels to see whether higher fertilizer would influence emf colorization
of roots. We obtained media without added fertilizer, asked the supplier
the quantities of added fertilizer per litre and then added this back in
at 25, 50, 75 and 100\% of their usual amounts.

We had a limited selection of emf at the time (due to other commitments
in production) and therefore only used heb and pax with birch, sui with
pine and sui heb and pax with sitka. Note that there was some query with
the heb sequence and on reBLAST it came back as pax - so this experiment
may be limited sitka/pax, sitka/sui, birch/pax, pine/sui. Further - note
that this is not a trial of responses due to different strains in the
library as we would need to trial all strains of interest at the same
time.

We also wanted to look at any pellet effect and therefore included the
use of an uninoculated blank pellet as well as a no pellet control.

There were 40 trees in each treatment.

Trees were miniplugs planted in early January.

This trial looked at:

\begin{enumerate}
\def\labelenumi{\arabic{enumi})}
\item
  Whether higher fertilizer levels would affect emf colonization rates
  on the roots (as quantified by change in height)
\item
  Any effect of pellet as a fertilizer
\item
  Any difference in growth of birch/pine/sitka with emf compared to
  control
\item
  Any difference in growth for birch and sitka between emf (no rigorous)
\end{enumerate}

Data collected in this run - initial height, final height, mortality

Treatments: Control - no pellet (co), Control with blank pellet (co),
heb, pax, sui

Trees, \emph{Betula pendula} (BP), \emph{Pinus sylvestris} (PS),
\emph{Picea sitchensis} (SITKA)

Data munging:

Initial height data was collected as frequencies, e.g.~number of sitka
of 2-3cm, number of 3-4cm, partly due to time constraints measuring
individual trees (PG set up this run alone) but in retrospect, since
they are so small, this might not be a bad way of recording them. This
initial data must first be reshaped into columns of heights.

initial := long\_df\_i

Final heights also need some re-shaping to a long df, but data was
recorded as height per tree

final := long\_df\_f

Final part of reshaping combines initial and final data, adds the time
point

long\_df\_i \& long\_df\_f:= combined\_df

Finally, change in height is calculated as the average of each treatment
at each fert level

merged\_df = average change in heights grouped by tree/emf/fert

Inital exploration - box plots of change in heights

\includegraphics{Run1_files/figure-latex/unnamed-chunk-2-1.pdf}
\includegraphics{Run1_files/figure-latex/unnamed-chunk-2-2.pdf}
\includegraphics{Run1_files/figure-latex/unnamed-chunk-2-3.pdf} Initial
exploration suggests birch and sitka had greater change in height than
controls, but this is less clear for pine. The blank pellet does not
appear to have had any affect.

\begin{enumerate}
\def\labelenumi{\arabic{enumi})}
\tightlist
\item
  Does fertilizer affect growth?
\end{enumerate}

To see whether there is a relationship between change in height and
fertilizer levels, use linear model

\includegraphics{Run1_files/figure-latex/unnamed-chunk-3-1.pdf} The
linear model suggests no correlation between fertilizer and change in
height. Although for cb/BP treatment the slope and R\textsuperscript{2}
= 0.59 looks like a correlation, p = 0.128 suggest it is weak. Similarly
for pax/BP there is a weak correlation (p = 0.08). Any correlation is
not consistent between treatments.

The lack of correlation may be due to improper mixing of fertilizer back
in to the media. The added chemicals were granular and hence it was
impossible to ensure full homogenization. Therefore, this data may be
inconclusive.

The final part of this analysis would be to observe roots, but so far we
have not had time/people to do that. Therefore it could be that case
that physical colonization is different, but it has not translated in a
change in height.

Since the impact of fertilizer on growth appears non-significant - all
the data will be combined to create a larger data set such that each
treatment had 5 repeats.

\includegraphics{Run1_files/figure-latex/unnamed-chunk-4-1.pdf}
\includegraphics{Run1_files/figure-latex/unnamed-chunk-4-2.pdf}
\includegraphics{Run1_files/figure-latex/unnamed-chunk-4-3.pdf}

The plots combine the fertilizer treatments into a repeat, so that there
are 5 reps of each emf/tree treatment. The coloured points on the charts
show the average change in height for that fertilizer level. That is,
the average of 40 trees per fertilizer level, The box plots visualize
all data for each treatment combined, that is 200 trees in total per
combined treatment. The p values are t test compare to co

Birch shows change in height for pellet compared to controls. There is a
significant change in height of \textasciitilde3.75cm over the growing
period

Pine shows no significant difference in change in height. This could be
for several reasons. i) The pine at Elsoms is already colonized by
something, we need to examine the roots to see what it is - might be
highly colonized by a sui to start with. ii) With the trays on the
matting, roots had grown out of the trays and into the mats and were
highly intermixed. Difficult to say that these were separate treatments

Sitka shows significant (p = 0.0498 compared to co) but very small
change in height of around 1.5cm for sui, but not for pax or heb. This
may be because sui is better at promoting growth in these conditions.
Height change may be small because plants were harvested before the
typical late season flush Blank pellet had no affect in any treatment

Note that for all birch and sitka, the heb/pax response was the same,
further indicating that this heb could be a pax.

Also note - no difference between co and cb.

\end{document}
