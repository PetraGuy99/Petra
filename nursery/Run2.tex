% Options for packages loaded elsewhere
\PassOptionsToPackage{unicode}{hyperref}
\PassOptionsToPackage{hyphens}{url}
%
\documentclass[
]{article}
\usepackage{amsmath,amssymb}
\usepackage{iftex}
\ifPDFTeX
  \usepackage[T1]{fontenc}
  \usepackage[utf8]{inputenc}
  \usepackage{textcomp} % provide euro and other symbols
\else % if luatex or xetex
  \usepackage{unicode-math} % this also loads fontspec
  \defaultfontfeatures{Scale=MatchLowercase}
  \defaultfontfeatures[\rmfamily]{Ligatures=TeX,Scale=1}
\fi
\usepackage{lmodern}
\ifPDFTeX\else
  % xetex/luatex font selection
\fi
% Use upquote if available, for straight quotes in verbatim environments
\IfFileExists{upquote.sty}{\usepackage{upquote}}{}
\IfFileExists{microtype.sty}{% use microtype if available
  \usepackage[]{microtype}
  \UseMicrotypeSet[protrusion]{basicmath} % disable protrusion for tt fonts
}{}
\makeatletter
\@ifundefined{KOMAClassName}{% if non-KOMA class
  \IfFileExists{parskip.sty}{%
    \usepackage{parskip}
  }{% else
    \setlength{\parindent}{0pt}
    \setlength{\parskip}{6pt plus 2pt minus 1pt}}
}{% if KOMA class
  \KOMAoptions{parskip=half}}
\makeatother
\usepackage{xcolor}
\usepackage[margin=1in]{geometry}
\usepackage{longtable,booktabs,array}
\usepackage{calc} % for calculating minipage widths
% Correct order of tables after \paragraph or \subparagraph
\usepackage{etoolbox}
\makeatletter
\patchcmd\longtable{\par}{\if@noskipsec\mbox{}\fi\par}{}{}
\makeatother
% Allow footnotes in longtable head/foot
\IfFileExists{footnotehyper.sty}{\usepackage{footnotehyper}}{\usepackage{footnote}}
\makesavenoteenv{longtable}
\usepackage{graphicx}
\makeatletter
\def\maxwidth{\ifdim\Gin@nat@width>\linewidth\linewidth\else\Gin@nat@width\fi}
\def\maxheight{\ifdim\Gin@nat@height>\textheight\textheight\else\Gin@nat@height\fi}
\makeatother
% Scale images if necessary, so that they will not overflow the page
% margins by default, and it is still possible to overwrite the defaults
% using explicit options in \includegraphics[width, height, ...]{}
\setkeys{Gin}{width=\maxwidth,height=\maxheight,keepaspectratio}
% Set default figure placement to htbp
\makeatletter
\def\fps@figure{htbp}
\makeatother
\setlength{\emergencystretch}{3em} % prevent overfull lines
\providecommand{\tightlist}{%
  \setlength{\itemsep}{0pt}\setlength{\parskip}{0pt}}
\setcounter{secnumdepth}{-\maxdimen} % remove section numbering
\usepackage{booktabs}
\usepackage{longtable}
\usepackage{array}
\usepackage{multirow}
\usepackage{wrapfig}
\usepackage{float}
\usepackage{colortbl}
\usepackage{pdflscape}
\usepackage{tabu}
\usepackage{threeparttable}
\usepackage{threeparttablex}
\usepackage[normalem]{ulem}
\usepackage{makecell}
\usepackage{xcolor}
\ifLuaTeX
  \usepackage{selnolig}  % disable illegal ligatures
\fi
\usepackage{bookmark}
\IfFileExists{xurl.sty}{\usepackage{xurl}}{} % add URL line breaks if available
\urlstyle{same}
\hypersetup{
  pdftitle={Elsoms Run 2},
  pdfauthor={Petra Guy},
  hidelinks,
  pdfcreator={LaTeX via pandoc}}

\title{Elsoms Run 2}
\author{Petra Guy}
\date{2025-11-26}

\begin{document}
\maketitle

This analysis summarizes Elsoms run 2. Nursery Trials/Experiments/Run
1/Elsoms Run 2 Results

The first few charts look at oak mortality. The oaks were very mildewed
and I had almost decided to omit them from the experiment. However, I
did a mortality check first. Some trees had died immediately on
planting, and this number is deducted from the total number of trees per
tray. The mortality is then the number of trees which died
subsequently/this reduced starting number

\includegraphics{Run2_files/figure-latex/unnamed-chunk-2-1.pdf}

Fig caption: Control, trees which received no inoculum. Treatment, these
trees received liquid innoculum of 6 different emf, heb = \emph{Hebeloma
spp}, lac = \emph{Laccaria bicolor}, lact = \emph{Lactarius torminosus},
pax = \emph{Paxillus involutus}, scl = \emph{Scleroderma areolatum}, sui
= \emph{Suillus bovinus}

If the droughting is to be believed, and this is an issue, because it
was an oversight by Elsoms rather than a controlled experiment, then the
emf only had an impact on mortality in the droughting treatment. This
data may be useful to inform future runs - but perhaps used with caution
with customers. Next graph combines all the oaks ignoring the droughting
treatment.

\includegraphics{Run2_files/figure-latex/unnamed-chunk-3-1.pdf} Fig
caption: Control, trees which received no innoculum. Treatment, these
trees received liquid innoculum of 6 different emf, heb = \emph{Hebeloma
spp}, lac = \emph{Laccaria bicolor}, lact = \emph{Lactarius torminosus},
pax = \emph{Paxillus involutus}, scl = \emph{Scleroderma areolatum}, sui
= \emph{Suillus bovinus}

Above we ignore whether the trays were on the droughted or non-droughted
bench. Since some of the trees died immediately on the same day as
planting, these were removed from the analysis as this was not related
to treatment. each group then has slightly different starting number of
trees and mortality is calculated as number of trees which subsequently
died/(80-number which died immediately)

Fig caption: Control, trees which received no innoculum. Treatment,
these trees received liquid innoculum of 6 different emf, heb =
\emph{Hebeloma spp}, lac = \emph{Laccaria bicolor}, lact =
\emph{Lactarius torminosus}, pax = \emph{Paxillus involutus}, scl =
\emph{Scleroderma areolatum}, sui = \emph{Suillus bovinus}. The stars
represent the significance of a Welch's t test with ** p \textless=
0.01, *** p \textless= 0.001. Applied emf gives a significant change in
height for Lactarius, Paxillus and Scleroderma. Note that we do not
expect a difference for Suillus, this treatment included partially as
cross check and partially to keep teratments equal across all tree
species tested. Seeing an effect for Suillus may have alerted us to some
other unmeasured effect

Here is above splitting into drought and non-drought

\includegraphics{Run2_files/figure-latex/unnamed-chunk-4-1.pdf} Fig
caption: Control, trees which received no innoculum. Treatment, these
trees received liquid innoculum of 6 different emf, heb = \emph{Hebeloma
spp}, lac = \emph{Laccaria bicolor}, lact = \emph{Lactarius torminosus},
pax = \emph{Paxillus involutus}, scl = \emph{Scleroderma areolatum}, sui
= \emph{Suillus bovinus}. The stars represent the significance of a
Welch's t test with ** p \textless= 0.01, *** p \textless= 0.001.

When we also split out the drought and non drought we have the same
pattern of lactarius, paxillus and scleroderma giving significant
changes in height, so whilst the droughting may influence mortality, it
does not appear to influence change in height.

Cleaner here to run a lm with droughting as an effect as multipl
pairwise t tests can inflate type 1 errors

\includegraphics{Run2_files/figure-latex/unnamed-chunk-5-1.pdf} Linear
model, main effects are fungi and drought, interaction effect of drought
(i,e, how each fungi responds differently under drought vs non drought).
Estimated marginal means (lm\_model \textasciitilde{} Fungi \textbar{}
Drought ) - this is delta H of lact for each drought level, estimated by
the model - not just the raw average. Contrast(emm) compares model
adjusted mean of lact under drought to model adjusted mean of control
under drought - similarly for non-drought.

\includegraphics{Run2_files/figure-latex/unnamed-chunk-6-1.pdf}

\includegraphics{Run2_files/figure-latex/unnamed-chunk-7-1.pdf}

\begin{longtable}[]{@{}lrrrr@{}}
\caption{Summary of Delta H by Fungal Treatment}\tabularnewline
\toprule\noalign{}
Fungi & n & Mean Delta H (cm) & SD (cm) & Variance \\
\midrule\noalign{}
\endfirsthead
\toprule\noalign{}
Fungi & n & Mean Delta H (cm) & SD (cm) & Variance \\
\midrule\noalign{}
\endhead
\bottomrule\noalign{}
\endlastfoot
control & 38 & 2.96 & 2.90 & 8.41 \\
heb & 69 & 4.24 & 3.81 & 14.50 \\
lac & 60 & 3.44 & 2.78 & 7.75 \\
lact & 71 & 5.64 & 4.51 & 20.36 \\
pax & 66 & 4.80 & 3.64 & 13.27 \\
scl & 67 & 4.93 & 3.59 & 12.88 \\
sui & 60 & 3.00 & 2.57 & 6.61 \\
\end{longtable}

\includegraphics{Run2_files/figure-latex/unnamed-chunk-8-1.pdf}

\end{document}
